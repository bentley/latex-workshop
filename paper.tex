\documentclass[12pt]{article}

\begin{document}

\linespread{1.5}

\author{Anthony J. Bentley}
\title{Melding Carbon Nanotubes and DNA\\
ECE 371---Nanotechnology Paper}
\maketitle

\newpage

\tableofcontents

\newpage

\section*{Introduction}
\addcontentsline{toc}{section}{Introduction}

\looseness +1 Carbon nanotubes are arrangements of carbon atoms in a cylindrical structure. Because of their highly regular geometry and interesting chemical and electromechanical properties, they have many uses in industry and research.

DNA is a biological macromolecule that occurs naturally in living organisms. Due to its importance to life, it has frequently been the subject of study and experimentation.

Carbon nanotubes and DNA both exist at a similar (nano) scale. As a result, research has been ongoing to discover ways to use them together in order to solve specific problems.

\section{Carbon nanotube separation via DNA}

Applications of carbon nanotubes sometimes require manipulation of single nanotubes, moving or shaping them in particular ways. However, carbon nanotubes can be difficult to separate. It is extremely rare for solids to be uniform, neat, and similarly sized, especially at the nano scale. As a result, carbon nanotubes have variations in length and mass---in a word, they are polydisperse.

Even worse, carbon nanotubes have a tendency to stay in clumps, because of binding energies. This is unavoidable, and leads to unfortunate difficulties in manipulating individual nanotubes.

Work has been done to separate carbon nanotubes despite these problems. Research published in molecular modeling has shown that single-stranded DNA will tend to wrap around a carbon nanotube, with a binding energy close to that of carbon to carbon. Experimental results show that when sonically agitated in the presence of DNA, clumps of carbon nanotubes will separate, and become individually dispersed.\cite{zheng03}

Other methods involving chromatography also allow purifying and then sorting individual nanotubes by length.\cite{doi:10.1021/ac0508954}

\section{DNA-wrapped carbon nanotubes}

\looseness +1 Once they have been separated, wrapping carbon nanotubes in single-stranded DNA will result in new effects and characteristics.\cite{doi:10.1021/jp204017u}\cite{enyashin07} One existing method uses the DNA to make carbon nanotubes more soluble in water. The resulting melded DNA--carbon nanotube is more useful in applications as a biosensor.\cite{C1AN15179G}

The characteristics of the nanocomposite depend on the sequence of the DNA used.\cite{Zheng28112003} This has led to speculation over the possibility of manufacturing DNA specifically for the purpose of creating particular quantum effects. In the meantime, some researchers have been compiling ``libraries'' of DNA in order to purify carbon nanotubes of specific chiralities.\cite{tu09}

\section{DNA sequencing via carbon nanotubes}

DNA translocation is a process that ``threads'' a strand of DNA through a nano-scale pore.\cite{ISI:000168045700055} This pore can be the end of a carbon nanotube. When a strand of DNA is drawn through the tube, the tube's electrical properties (such as ionic current) change detectably, but not significantly.

\looseness +1 Recent work has developed a method that greatly amplifies the charge on the DNA itself, allowing practical electrical measurements at the molecular level.\cite{ISI:000273395400030} This can be used for rapid DNA sequencing---the electrical characteristics would be strong enough that existing equipment could distinguish between individual nucleotide bases (the ``DNA letters'' A, T, G, and~C).

\section*{Conclusion}
\addcontentsline{toc}{section}{Conclusion}

There have been many exciting developments in nanocomposites and DNA--carbon nanotube melding. What is even more intriguing is the thought of what the future holds. The scientific discoveries described here will hopefully spur amazing advances in science and technology, resulting in better lives for all of mankind.

\bibliographystyle{plain}
\bibliography{paper}

\end{document}
