\documentclass{beamer}
\usepackage{amsmath}

\title{Introduction to \LaTeX}
\author{Anthony J. Bentley}

\institute{UNM--IEEE}

\begin{document}
\begin{frame}
\titlepage
\end{frame}

\begin{frame}{What is \TeX?}
\begin{itemize}
\pause\item{Invented by Donald Knuth in 1978}
\pause\item{A typesetter---converts plain text to a printer-ready document}
\pause\item{\LaTeX\ came later, a set of document macros instead of fine-grained typesetting}
\end{itemize}
\end{frame}

\begin{frame}{Why use \LaTeX?}
\begin{itemize}
\pause\item{It makes things easier}
\pause\item{Other people in industry and academia use it}
\pause\item{It's available for free, legally}
\end{itemize}

\end{frame}

\begin{frame}{How to install}
Lots of popular \LaTeX distributions.

A popular version for Windows is MiK\TeX.

Download it here:

\url{http://miktex.org/portable/about}

Then extract it to your flash drive, and run it anywhere
\end{frame}

\begin{frame}[fragile]{Demonstration}

Double-click ``miktex-portable.cmd'',

then right-click the taskbar icon and open a command prompt.

\pause

\begin{verbatim}
\documentclass{article}

\begin{document}
Hello World!
\end{document}
\end{verbatim}

\pause
Save as a text file, then ``pdflatex filename.tex''

\end{frame}

\begin{frame}[fragile]{More basics}

\begin{verbatim}
This is printed text. % This is a comment, which is hidden.

Spaces don't matter. These two lines look the same.

Spaces don't  matter.   These    two     lines
look
the
same.
\end{verbatim}
\end{frame}

\begin{frame}[fragile]{Formatting}

\LaTeX\ provides sections, subsections, lists, tables\ldots

\begin{verbatim}
\section{Introduction}
Welcome to English 219. I'm taking this for two reasons:
\begin{itemize}
\item{It's required.}
\item{It's required.}
\end{itemize}
\end{verbatim}

\end{frame}

\begin{frame}[fragile]{Math}

\ldots and math!

\begin{verbatim}
\begin{equation}
1 + 1 = 2
\end{equation}
\end{verbatim}
\begin{equation}
1 + 1 = 2
\end{equation}

\pause
\begin{verbatim}
\begin{equation}
e = mc^2
\end{equation}
\end{verbatim}
\begin{equation}
e = mc^2
\end{equation}

\end{frame}

\begin{frame}[fragile]{Math}

\begin{verbatim}
\begin{equation}
\cos \theta = \pm \sqrt{ 1 - \sin^2 \theta }
\end{equation}
\end{verbatim}
\begin{equation}
\cos \theta = \pm \sqrt{ 1 - \sin^2 \theta }
\end{equation}

\begin{verbatim}
\begin{equation}
x = \frac{-b\pm\sqrt{b^2-4ac}}{2a}
\end{equation}
\end{verbatim}
\begin{equation}
x = \frac{-b\pm\sqrt{b^2-4ac}}{2a}
\end{equation}

\end{frame}

\begin{frame}[fragile]{Math}

The ``amsmath'' package (from the American Mathematical Society) provides things like matrices.

After \verb#\documentclass{article}#, put \verb#\usepackage{amsmath}#.
\begin{columns}
\begin{column}{0.5\textwidth}
\begin{verbatim}
\begin{equation}
x = \frac{
\left|\begin{matrix}
e & b\\ f & d
\end{matrix}\right|
}{
\left|\begin{matrix}
a & b\\ c & d
\end{matrix}\right|
}
\end{equation}
\end{verbatim}
\end{column}

\begin{column}{0.3\textwidth}
\begin{equation}
x = \frac{
\left|
\begin{matrix}
e & b\\
f & d
\end{matrix}
\right|
}{
\left|
\begin{matrix}
a & b\\
c & d
\end{matrix}
\right|
}
\end{equation}
\end{column}
\end{columns}

\end{frame}

\begin{frame}{Useful links}
A good \LaTeX manual

\url{http://www.eng.cam.ac.uk/help/tpl/textprocessing/ltxprimer-1.0.pdf}
\pause
\vspace{1em}

Handwriting recognition

\url{http://webdemo.visionobjects.com/equation.html}

\url{http://detexify.kirelabs.org/classify.html}
\pause

\vspace{1em}
These slides, and some more example papers:

\url{https://github.com/bentley/latex-workshop/}
\end{frame}
\end{document}
